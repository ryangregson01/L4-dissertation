% REMEMBER: You must not plagiarise anything in your report. Be extremely careful.

\documentclass{l4proj}

    
%
% put any additional packages here
%

\begin{document}

%==============================================================================
%% METADATA
\title{Level 4 Individual Project}
\author{Ryan Gregson}
\date{February, 2023}

\maketitle

%==============================================================================
%% ABSTRACT
\begin{abstract}
    Every abstract follows a similar pattern. Motivate; set aims; describe work; explain results.
    \vskip 0.5em
    ``XYZ is bad. This project investigated ABC to determine if it was better. 
    ABC used XXX and YYY to implement ZZZ. This is particularly interesting as XXX and YYY have
    never been used together. It was found that  
    ABC was 20\% better than XYZ, though it caused rabies in half of subjects.''
\end{abstract}

%==============================================================================

% EDUCATION REUSE CONSENT FORM
% If you consent to your project being shown to future students for educational purposes
% then insert your name and the date below to  sign the education use form that appears in the front of the document. 
% You must explicitly give consent if you wish to do so.
% If you sign, your project may be included in the Hall of Fame if it scores particularly highly.
%
% Please note that you are under no obligation to sign 
% this declaration, but doing so would help future students.
%
%\def\consentname {My Name} % your full name
%\def\consentdate {20 March 2018} % the date you agree
%
\educationalconsent


%==============================================================================
\tableofcontents

%==============================================================================
%% Notes on formatting
%==============================================================================
% The first page, abstract and table of contents are numbered using Roman numerals and are not
% included in the page count. 
%
% From now on pages are numbered
% using Arabic numerals. Therefore, immediately after the first call to \chapter we need the call
% \pagenumbering{arabic} and this should be called once only in the document. 
%
% Do not alter the bibliography style.
%
% The first Chapter should then be on page 1. You are allowed 40 pages for a 40 credit project and 30 pages for a 
% 20 credit report. This includes everything numbered in Arabic numerals (excluding front matter) up
% to but excluding the appendices and bibliography.
%
% You must not alter text size (it is currently 10pt) or alter margins or spacing.
%
%
%==================================================================================================================================
%
% IMPORTANT
% The chapter headings here are **suggestions**. You don't have to follow this model if
% it doesn't fit your project. Every project should have an introduction and conclusion,
% however. 
%
%==================================================================================================================================
\chapter{Introduction}

% reset page numbering. Don't remove this!
\pagenumbering{arabic} 

\section{Motivation}

Disordered proteins cause neurodegenerative diseases such as Alzheimer’s, Parkinson’s and Huntington’s. These intrinsically disordered proteins do not have a fixed or ordered three-dimensional structure and there are significant differences between the amino acid sequences of intrinsically disordered proteins (IDPs) / intrinsically disordered regions (IDRs) compared to structured globular (functional) proteins \cite{disordered_proteins_diseases_paper}. The structural order of these sequences can be manually labelled from experiments with x-ray crystallographic analysis \cite{idp_wiki}. These experiments can be difficult and the number of newly occurring proteins submitted to the Protein Data Bank (PDB) \cite{pdb} each year makes it difficult to manually experiment with each one, meaning many proteins’ disorder structure have not been assessed. We can use bioinformatic tools to label the disordered regions in these protein sequences by allowing the tool to learn the differences between ordered and disordered amino acid sequences. With more labelled sequences, doctors and researchers can understand more about disordered protein sequences. A better understanding of these regions helps research into the development of drugs to counteract diseases and contributes to keeping a well-maintained knowledge database of proteins.  

Deep learning approaches have become incredibly popular recently, and these approaches currently dominate classification tasks. Deep learning architectures are prevalent in classifying the IDRs in protein sequences, where state of the art predictors such as SPOT-Disorder \cite{spot_disorder_2} have online protein disorder prediction servers for others to use their deep learning model.

\section{General problem and aim}

Protein sequences are a string of amino acid characters. The problem we are attempting to solve is to see if we can use these sequences to predict whether each amino acid is ordered or disordered. We will produce deep neural networks which will take in protein sequences and predict the disordered regions within them. 

The aim of this project is to explore how different deep learning approaches and architectures perform at predicting protein disorder. To assess these deep learning methods, we will: 
\begin{itemize}
    \item Discuss current solutions to protein disorder prediction.
    \item Design and implement different deep neural networks, such as a convolutional neural network (CNN) and recurrent neural network (RNN). 
    \item Evaluate our implemented neural networks using standard metrics and the CASP dataset \cite{casp}, which is used for evaluating protein disorder prediction models. 
\end{itemize}
An aim alongside comparing the deep neural networks is to create a server, where a user can input a protein sequence via a webpage form and see the predicted disordered labels. This will result in individuals being able to interact with my developed solution easily. 


%==================================================================================================================================
\chapter{Background}
\label{chap:background}

\section{Biology overview}

\subsection{What is a protein?}

Proteins are a linear sequence of amino acids \cite{prot_struct_lib}. An example of this sequence is the protein, Transmembrane protein 218, with a linear sequence of 115 amino acids \textbf{figureA2RU14Sequence} \cite{prot_A2RU14}. There are twenty possible amino acids \cite{aa_wiki} protein sequences can be built from, and each of these amino acids have a unique side chain which have different structural properties. Different proteins have different sequences of amino acids. The amino acids in the sequence will ‘interact with each other to form a well-defined, folded protein’ \cite{protein_folding}. The folded protein with have a three-dimensional structure, and this three-dimensional structure determines the protein's function \cite{bbc_bitesize}. Function can be from supporting frameworks inside cells, catalysing biological reactions, communication between different parts of the body and used for function inside the immune system \cite{bbc_bitesize}. Therefore, it is important that the folding of proteins is done correctly as misfolded proteins will lose their structure and function, which can lead to diseases. 

\subsection{Intrinsically disordered proteins (IDPs)}
\label{chap:background sec:IDPs}

IDPs are different to structured proteins. These unstructured, disordered proteins are suggested to cause a number of diseases. Their amino acid sequences differ and contain a “low content of bulky hydrophobic amino acids and a high proportion of polar and charged amino acids” \cite{idp_wiki}. In comparison to structured proteins that fold correctly these sequences cannot fold into stable globular proteins, meaning that the protein lacks an ordered three-dimensional structure. The root cause of this lack of structure, causing disorder, is the proteins’ primary structure: the amino acid sequence. When disordered amino acids are grouped, this is labelled as an intrinsically disordered region (IDR) and is where the misfolding occurs. Therefore, making predictions about the probability that an amino acid residue is ordered or disordered within a protein sequence will help determine where this incorrect folding will occur. This means by focusing on the amino acids’ structure we can determine the outcome of the protein functioning correctly. 

A current problem is the lack of labelled data about the structure of protein sequences as determining these via experiments such as nuclear magnetic resonance (NMR) and small-angle X-ray scattering (SAXS) is very costly and time-consuming \cite{disordered_prot_genus_camelus}. To tackle this challenge, bioinformatic tools have been created for protein sequence structure labelling. These tools are often accessible via a web server, such as the DISOPRED server \cite{disopred3_paper}, which uses machine learning methods to make predictions. This project will produce a similar tool for the prediction of protein disorder after experimentation with different deep learning architectures and approaches. 

\section{Deep learning overview}

\subsection{Deep Neural Networks}

Deep neural networks are made up of stacked neural network layers. As these layers are stacked, the output of one layer will be used as input to the next layer in the network. Each of these layers performs feature extraction on their input. Each layer consists of an arbitrary number of neurons, each with their own parameter weight which is learnt during training. During training, inputs are classified by the model during a forward pass through the model. These predictions are compared to the true value using a loss function. Through the process of backpropagation \cite{Goodfellow-et-al-2016}, the outcome of the loss function is used and the parameter weights for each neuron within the network are changed in an aim to improve the models' predictions. To complete this training step, an optimisation method will update each layer of the network, changing the weights of the model as proposed by backpropagation. This training procedure is done with each item in the training set and for multiple iterations of the entire training set, to solve the optimisation problem by minimising the loss and finding optimum weights for each neuron. The machine learning framework, PyTorch \cite{pytorch}, provides the functionality to build neural network layers that extract features, use a loss function with an implemented backpropagation method and create an optimiser to update the model/classifier during training. PyTorch also provides efficient tensor computations and can utilise GPUs while training. 

\subsection{Convolutional Neural Networks (CNNs)}

A CNN is a type of neural network that uses convolution kernels to extract features from an input. Convolution kernels can be 1D kernels of size 1xK primarily used for sequence or series data, or 2D kernels of size NxM primarily used for classifying images. These kernels move across a grid shaped input to check if a feature is present. These kernels will be able to slide over a window of amino acids, which is like the sliding window-based approach that DISOPRED \ref{disopred} uses. With our protein sequences, it is intuitive to see that a kernel of 1xK will be able to slide along a [1, sequence length] grid, where we can treat each amino acid as a different channel to use 1D convolution kernels. We can also treat protein sequences as images, where the amino acids are the height of the image, and the sequence length is the width of the image. These are discussed further in our design \ref{chap:design}. 

In classification tasks convolution kernels are translation-invariant and require few parameters. These are beneficial as this means it does not matter where the location of the disordered region exists in the sequence. Less parameters means it can be trained with smaller training sets, which is good as the labelled disorder datasets from Disprot \cite{disprot} is small in comparison to other deep learning tasks. However, papers such as DISOPRED3 state further annotated training labels will be beneficial to prediction models \cite{disopred3_paper}.

During most classification tasks, input of the same shape is passed to the CNN. However, with our protein sequence data the lengths of our sequences range from 19 \cite{prot_P0C8E0} to 34350 \cite{prot_Q8WZ42}, therefore we would need to unnecessarily pad many sequences with lots of redundant blank values. Therefore, we have taken an approach to use a fully convolutional network (FCN). Modern classification networks have been adapted to fully convolutional networks by Shelhamer and Long \cite{fcn_seg}, demonstrating that classification can be done using inputs of arbitrary size. Therefore, for these varying sequence length proteins, an FCN will be suited to handling these inputs. 



\newpage
What did other people do, and how is it relevant to what you want to do?
\section{Guidance}
\begin{itemize}    
    \item
      Don't give a laundry list of references.
    \item
      Tie everything you say to your problem.
    \item
      Present an argument.
    \item Think critically; weigh up the contribution of the background and put it in context.    
    \item
      \textbf{Don't write a tutorial}; provide background and cite
      references for further information.
\end{itemize}

%==================================================================================================================================
\chapter{Analysis/Requirements}

\section{General deep learning problem}

In our discussion of intrinsically disordered proteins \ref{chap:background sec:IDPs}, we demonstrate that the sequences of these proteins are different compared to ordered protein sequences. If we evaluate which amino acids appear within specific regions and where they appear, we can draw conclusions about where these disordered regions are. Therefore, we will use a supervised learning approach with our deep neural network, like current approaches \ref{sec:current approaches}, by using these amino acid sequences alongside their labelled disordered regions.  

\section{Our dataset}

As the identification of these disordered regions requires complex experimentation \ref{chap:background}, we must use a curated database of disordered proteins. The Disprot database \cite{disprot} has been used for a variety of machine learning tasks \cite{disopred3_paper} \cite{Xue:10} \cite{Mizianty:13} \cite{Hanson:16} involving disordered proteins. This database is regarded as the “major repository of manually curated data for intrinsically disordered proteins” \cite{Quaglia:22} and is widely used by researchers studying disordered proteins. Disprot stores information about the positions of IDRs in sequences and uses a UniProtKB accession number to identify the protein. This identifier will allow us to retrieve the full protein sequence from the UniProt database \cite{uniprot:22}, another manually curated database for proteins, to help us train our network. 

\section{Labelled data}

Compared to other problems solved by deep neural networks \cite{Jumper:21} \cite{Xie:17} \cite{Devlin:18}, the size of our manually labelled dataset is much smaller. Disprot only contains around 2500 protein sequences. However, we are not classifying an entire sequence as either ordered or disordered, we are classifying a single amino acid within the sequence as ordered or disordered \textbf{figure{AAgroundtruthlabel}}. This is because we know the region of the protein can be disordered due to the characteristics of the grouped amino acids. This gives us sufficient labelled data. Using these labelled sequences as our ground truth, we can utilise our deep neural networks to recognise patterns and relationships about the amino acid sequence that causes intrinsic disorder in proteins through our supervised learning approach. 

\section{Approaches}

Machine learning algorithms cannot operate on our protein sequences directly. Therefore, to ensure our model is equipped to handle the various amino acids, we will consider them as distinct categories of data and employ feature encoding techniques to generate a suitable numerical input \cite{Kumar:20}.

\subsection{One-hot}

Our first feature encoding approach is one-hot encoding \cite{One-hot:wiki} \cite{Fawcett:21}. We can treat each amino acid as a vector by using the one-hot encoding technique. Our vectors will have a size of 20 because of the twenty possible amino acids, therefore as these feature vectors are not excessively large, this one-hot approach is suitable. This has let us transform a raw amino acid sequence into a feature matrix, of shape [20, sequence length], which now represents the sequence.

\subsection{Position-Specific Scoring Matrix (PSSM)}

Our other feature encoding approach is to generate a position-specific scoring matrix (PSSM) \cite{PSSM:wiki}. This technique is used in many protein structure problems, such as \cite{McGuffin:00} \cite{Wang:17} and used within current approaches previously discussed \ref{sec: current approaches}. A PSSM is a “matrix that involves information about the probability of amino acids or nucleotides occurrence in each position, which is derived from a multiple sequence alignment” \cite{Mohammadi:22}. These PSSMs provide an informative representation about the amino acids at positions within the protein sequences which will be helpful for our network to identify patterns and relationships within protein sequences. Furthermore, as PSSMs are also of shape [20, sequence length], the neural network models will be able to take both one-hot encoded vectors and PSSMs as input without any changes, allowing us to compare these different feature encoding methods.

\section{Architectures}

\section{Evaluation metrics}
With our trained deep neural network, we will evaluate its predictions using similar approaches by the literature. As this is a binary classification task classifying amino acids as either ordered or disordered we can define our outcomes as: 

\begin{itemize}    
    \item True positive (TP): When the model correctly predicts that a disordered residue is disordered.
    \item True negative (TN): When the model correctly predicts that an ordered residue is ordered.
    \item False positive (FP): When the model incorrectly predicts that an ordered residue is disordered.
    \item False negative (FN): When the model incorrectly predicts that a disordered residue is ordered. \cite{google_TP}
\end{itemize}

By computing these outcomes, we can evaluate the accuracy, precision, recall (sensitivity) and specificity of our model. These metrics are widely used in various machine learning tasks as standard performance measures \cite{BC:wiki}. We can also calculate the F1-score which is the harmonic mean of the precision and recall and is another useful measure of how accurate our model is. Furthermore, as our dataset has a bias of ordered labels, we will consider the Matthews correlation coefficient (MCC), which is a more reliable measure when dealing with an imbalanced or disproportionate dataset \cite{Chicco:20}.

With these evaluation metrics we will be able to compare our approaches and architectures against each other. We can also compare our models to known current approaches [ref current approaches], using the CASP datasets which are used for benchmarking protein disorder prediction models \cite{casp}.

\section{Django server}

In addition to evaluating deep learning models, it is necessary to create a Django-based server that enables users to submit a protein sequence and obtain a comprehensible representation of the disordered regions of the protein. From the literature, papers proposing protein disorder prediction tools also often set up an online server so their tool can be used \cite{DISOPRED:server} \cite{Metapredict:server}. These protein prediction servers are important as they allow researchers to easily access these tools, allowing them to understand the structure and function of proteins which is useful for protein engineering \cite{prot_engineering:wiki} \cite{Engvist:19}. 

A brief analysis of the server’s requirements follows following the MoSCoW prioritisation method \cite{moscow}: 
\subsection{Functional requirements}

\begin{itemize}    
    \item Must have: The server will have a web interface, so it is easily accessible for use by researchers.
    \item Must have: Users can submit a single protein sequence via a form.
    \item Must have: The server can make a prediction on a protein sequence using a deep neural network. This model will be chosen from our evaluation.
    \item Should have: The web interface will have a clear visualisation displaying the sequence to the user with clearly identified IDRs.
    \item Could have: The web interface will have a graphical representation of the identified IDRs.
    \item Will not have this time: The visualisation of the disordered sequence cannot be saved.
    \item Will not have this time: The server will not allow multiple sequences to be inputted in a single form entry.
\end{itemize}

\subsection{Non-functional requirements of our server}

\begin{itemize}    
    \item Must have: The web interface will be intuitive and easy to use. Both when entering information in via the form and interpreting the outputted sequence with highlighted disordered regions.
    \item Should have: The web application should be containerised to follow best practices and be easily portable.
    \item Could have: The Django server could interact with an external database for efficient lookup of previously entered sequences and their representations.
    \item Will not have this time: The web page will not be hosted and accessible via a searchable domain by the conclusion of the project.
\end{itemize}


\newpage
What is the problem that you want to solve, and how did you arrive at it?
\section{Guidance}
Make it clear how you derived the constrained form of your problem via a clear and logical process. 

%==================================================================================================================================
\chapter{Design}
\label{chap:design}
How is this problem to be approached, without reference to specific implementation details? 
\section{Guidance}
Design should cover the abstract design in such a way that someone else might be able to do what you did, but with a different language or library or tool.

%==================================================================================================================================
\chapter{Implementation}
What did you do to implement this idea, and what technical achievements did you make?
\section{Guidance}
You can't talk about everything. Cover the high level first, then cover important, relevant or impressive details.



\section{General points}

These points apply to the whole dissertation, not just this chapter.



\subsection{Figures}
\emph{Always} refer to figures included, like Figure \ref{fig:relu}, in the body of the text. Include full, explanatory captions and make sure the figures look good on the page.
You may include multiple figures in one float, as in Figure \ref{fig:synthetic}, using \texttt{subcaption}, which is enabled in the template.



% Figures are important. Use them well.
\begin{figure}
    \centering
    \includegraphics[width=0.5\linewidth]{images/relu.pdf}    

    \caption{In figure captions, explain what the reader is looking at: ``A schematic of the rectifying linear unit, where $a$ is the output amplitude,
    $d$ is a configurable dead-zone, and $Z_j$ is the input signal'', as well as why the reader is looking at this: 
    ``It is notable that there is no activation \emph{at all} below 0, which explains our initial results.'' 
    \textbf{Use vector image formats (.pdf) where possible}. Size figures appropriately, and do not make them over-large or too small to read.
    }

    % use the notation fig:name to cross reference a figure
    \label{fig:relu} 
\end{figure}


\begin{figure}
    \centering
    \begin{subfigure}[b]{0.45\textwidth}
        \includegraphics[width=\textwidth]{images/synthetic.png}
        \caption{Synthetic image, black on white.}
        \label{fig:syn1}
    \end{subfigure}
    ~ %add desired spacing between images, e. g. ~, \quad, \qquad, \hfill etc. 
      %(or a blank line to force the subfigure onto a new line)
    \begin{subfigure}[b]{0.45\textwidth}
        \includegraphics[width=\textwidth]{images/synthetic_2.png}
        \caption{Synthetic image, white on black.}
        \label{fig:syn2}
    \end{subfigure}
    ~ %add desired spacing between images, e. g. ~, \quad, \qquad, \hfill etc. 
    %(or a blank line to force the subfigure onto a new line)    
    \caption{Synthetic test images for edge detection algorithms. \subref{fig:syn1} shows various gray levels that require an adaptive algorithm. \subref{fig:syn2}
    shows more challenging edge detection tests that have crossing lines. Fusing these into full segments typically requires algorithms like the Hough transform.
    This is an example of using subfigures, with \texttt{subref}s in the caption.
    }\label{fig:synthetic}
\end{figure}

\clearpage

\subsection{Equations}

Equations should be typeset correctly and precisely. Make sure you get parenthesis sizing correct, and punctuate equations correctly 
(the comma is important and goes \textit{inside} the equation block). Explain any symbols used clearly if not defined earlier. 

For example, we might define:
\begin{equation}
    \hat{f}(\xi) = \frac{1}{2}\left[ \int_{-\infty}^{\infty} f(x) e^{2\pi i x \xi} \right],
\end{equation}    
where $\hat{f}(\xi)$ is the Fourier transform of the time domain signal $f(x)$.

\subsection{Algorithms}
Algorithms can be set using \texttt{algorithm2e}, as in Algorithm \ref{alg:metropolis}.

% NOTE: line ends are denoted by \; in algorithm2e
\begin{algorithm}
    \DontPrintSemicolon
    \KwData{$f_X(x)$, a probability density function returing the density at $x$.\; $\sigma$ a standard deviation specifying the spread of the proposal distribution.\;
    $x_0$, an initial starting condition.}
    \KwResult{$s=[x_1, x_2, \dots, x_n]$, $n$ samples approximately drawn from a distribution with PDF $f_X(x)$.}
    \Begin{
        $s \longleftarrow []$\;
        $p \longleftarrow f_X(x)$\;
        $i \longleftarrow 0$\;
        \While{$i < n$}
        {
            $x^\prime \longleftarrow \mathcal{N}(x, \sigma^2)$\;
            $p^\prime \longleftarrow f_X(x^\prime)$\;
            $a \longleftarrow \frac{p^\prime}{p}$\;
            $r \longleftarrow U(0,1)$\;
            \If{$r<a$}
            {
                $x \longleftarrow x^\prime$\;
                $p \longleftarrow f_X(x)$\;
                $i \longleftarrow i+1$\;
                append $x$ to $s$\;
            }
        }
    }
    
\caption{The Metropolis-Hastings MCMC algorithm for drawing samples from arbitrary probability distributions, 
specialised for normal proposal distributions $q(x^\prime|x) = \mathcal{N}(x, \sigma^2)$. The symmetry of the normal distribution means the acceptance rule takes the simplified form.}\label{alg:metropolis}
\end{algorithm}

\subsection{Tables}

If you need to include tables, like Table \ref{tab:operators}, use a tool like https://www.tablesgenerator.com/ to generate the table as it is
extremely tedious otherwise. 

\begin{table}[]
    \caption{The standard table of operators in Python, along with their functional equivalents from the \texttt{operator} package. Note that table
    captions go above the table, not below. Do not add additional rules/lines to tables. }\label{tab:operators}
    %\tt 
    \rowcolors{2}{}{gray!3}
    \begin{tabular}{@{}lll@{}}
    %\toprule
    \textbf{Operation}    & \textbf{Syntax}                & \textbf{Function}                            \\ %\midrule % optional rule for header
    Addition              & \texttt{a + b}                          & \texttt{add(a, b)}                                    \\
    Concatenation         & \texttt{seq1 + seq2}                    & \texttt{concat(seq1, seq2)}                           \\
    Containment Test      & \texttt{obj in seq}                     & \texttt{contains(seq, obj)}                           \\
    Division              & \texttt{a / b}                          & \texttt{div(a, b) }  \\
    Division              & \texttt{a / b}                          & \texttt{truediv(a, b) } \\
    Division              & \texttt{a // b}                         & \texttt{floordiv(a, b)}                               \\
    Bitwise And           & \texttt{a \& b}                         & \texttt{and\_(a, b)}                                  \\
    Bitwise Exclusive Or  & \texttt{a \textasciicircum b}           & \texttt{xor(a, b)}                                    \\
    Bitwise Inversion     & \texttt{$\sim$a}                        & \texttt{invert(a)}                                    \\
    Bitwise Or            & \texttt{a | b}                          & \texttt{or\_(a, b)}                                   \\
    Exponentiation        & \texttt{a ** b}                         & \texttt{pow(a, b)}                                    \\
    Identity              & \texttt{a is b}                         & \texttt{is\_(a, b)}                                   \\
    Identity              & \texttt{a is not b}                     & \texttt{is\_not(a, b)}                                \\
    Indexed Assignment    & \texttt{obj{[}k{]} = v}                 & \texttt{setitem(obj, k, v)}                           \\
    Indexed Deletion      & \texttt{del obj{[}k{]}}                 & \texttt{delitem(obj, k)}                              \\
    Indexing              & \texttt{obj{[}k{]}}                     & \texttt{getitem(obj, k)}                              \\
    Left Shift            & \texttt{a \textless{}\textless b}       & \texttt{lshift(a, b)}                                 \\
    Modulo                & \texttt{a \% b}                         & \texttt{mod(a, b)}                                    \\
    Multiplication        & \texttt{a * b}                          & \texttt{mul(a, b)}                                    \\
    Negation (Arithmetic) & \texttt{- a}                            & \texttt{neg(a)}                                       \\
    Negation (Logical)    & \texttt{not a}                          & \texttt{not\_(a)}                                     \\
    Positive              & \texttt{+ a}                            & \texttt{pos(a)}                                       \\
    Right Shift           & \texttt{a \textgreater{}\textgreater b} & \texttt{rshift(a, b)}                                 \\
    Sequence Repetition   & \texttt{seq * i}                        & \texttt{repeat(seq, i)}                               \\
    Slice Assignment      & \texttt{seq{[}i:j{]} = values}          & \texttt{setitem(seq, slice(i, j), values)}            \\
    Slice Deletion        & \texttt{del seq{[}i:j{]}}               & \texttt{delitem(seq, slice(i, j))}                    \\
    Slicing               & \texttt{seq{[}i:j{]}}                   & \texttt{getitem(seq, slice(i, j))}                    \\
    String Formatting     & \texttt{s \% obj}                       & \texttt{mod(s, obj)}                                  \\
    Subtraction           & \texttt{a - b}                          & \texttt{sub(a, b)}                                    \\
    Truth Test            & \texttt{obj}                            & \texttt{truth(obj)}                                   \\
    Ordering              & \texttt{a \textless b}                  & \texttt{lt(a, b)}                                     \\
    Ordering              & \texttt{a \textless{}= b}               & \texttt{le(a, b)}                                     \\
    % \bottomrule
    \end{tabular}
    \end{table}
\subsection{Code}

Avoid putting large blocks of code in the report (more than a page in one block, for example). Use syntax highlighting if possible, as in Listing \ref{lst:callahan}.

\begin{lstlisting}[language=python, float, caption={The algorithm for packing the $3\times 3$ outer-totalistic binary CA successor rule into a 
    $16\times 16\times 16\times 16$ 4 bit lookup table, running an equivalent, notionally 16-state $2\times 2$ CA.}, label=lst:callahan]
    def create_callahan_table(rule="b3s23"):
        """Generate the lookup table for the cells."""        
        s_table = np.zeros((16, 16, 16, 16), dtype=np.uint8)
        birth, survive = parse_rule(rule)

        # generate all 16 bit strings
        for iv in range(65536):
            bv = [(iv >> z) & 1 for z in range(16)]
            a, b, c, d, e, f, g, h, i, j, k, l, m, n, o, p = bv

            # compute next state of the inner 2x2
            nw = apply_rule(f, a, b, c, e, g, i, j, k)
            ne = apply_rule(g, b, c, d, f, h, j, k, l)
            sw = apply_rule(j, e, f, g, i, k, m, n, o)
            se = apply_rule(k, f, g, h, j, l, n, o, p)

            # compute the index of this 4x4
            nw_code = a | (b << 1) | (e << 2) | (f << 3)
            ne_code = c | (d << 1) | (g << 2) | (h << 3)
            sw_code = i | (j << 1) | (m << 2) | (n << 3)
            se_code = k | (l << 1) | (o << 2) | (p << 3)

            # compute the state for the 2x2
            next_code = nw | (ne << 1) | (sw << 2) | (se << 3)

            # get the 4x4 index, and write into the table
            s_table[nw_code, ne_code, sw_code, se_code] = next_code

        return s_table

\end{lstlisting}

%==================================================================================================================================
\chapter{Evaluation} 
How good is your solution? How well did you solve the general problem, and what evidence do you have to support that?

\section{Guidance}
\begin{itemize}
    \item
        Ask specific questions that address the general problem.
    \item
        Answer them with precise evidence (graphs, numbers, statistical
        analysis, qualitative analysis).
    \item
        Be fair and be scientific.
    \item
        The key thing is to show that you know how to evaluate your work, not
        that your work is the most amazing product ever.
\end{itemize}

\section{Evidence}
Make sure you present your evidence well. Use appropriate visualisations, reporting techniques and statistical analysis, as appropriate.

If you visualise, follow the basic rules, as illustrated in Figure \ref{fig:boxplot}:
\begin{itemize}
\item Label everything correctly (axis, title, units).
\item Caption thoroughly.
\item Reference in text.
\item \textbf{Include appropriate display of uncertainty (e.g. error bars, Box plot)}
\item Minimize clutter.
\end{itemize}

See the file \texttt{guide\_to\_visualising.pdf} for further information and guidance.

\begin{figure}
    \centering
    \includegraphics[width=1.0\linewidth]{images/boxplot_finger_distance.pdf}    

    \caption{Average number of fingers detected by the touch sensor at different heights above the surface, averaged over all gestures. Dashed lines indicate
    the true number of fingers present. The Box plots include bootstrapped uncertainty notches for the median. It is clear that the device is biased toward 
    undercounting fingers, particularly at higher $z$ distances.
    }

    % use the notation fig:name to cross reference a figure
    \label{fig:boxplot} 
\end{figure}


%==================================================================================================================================
\chapter{Conclusion}    
Summarise the whole project for a lazy reader who didn't read the rest (e.g. a prize-awarding committee).
\section{Guidance}
\begin{itemize}
    \item
        Summarise briefly and fairly.
    \item
        You should be addressing the general problem you introduced in the
        Introduction.        
    \item
        Include summary of concrete results (``the new compiler ran 2x
        faster'')
    \item
        Indicate what future work could be done, but remember: \textbf{you
        won't get credit for things you haven't done}.
\end{itemize}

%==================================================================================================================================
%
% 
%==================================================================================================================================
%  APPENDICES  

\begin{appendices}

\chapter{Appendices}

Typical inclusions in the appendices are:

\begin{itemize}
\item
  Copies of ethics approvals (required if obtained)
\item
  Copies of questionnaires etc. used to gather data from subjects.
\item
  Extensive tables or figures that are too bulky to fit in the main body of
  the report, particularly ones that are repetitive and summarised in the body.

\item Outline of the source code (e.g. directory structure), or other architecture documentation like class diagrams.

\item User manuals, and any guides to starting/running the software.

\end{itemize}

\textbf{Don't include your source code in the appendices}. It will be
submitted separately.

\end{appendices}

%==================================================================================================================================
%   BIBLIOGRAPHY   

% The bibliography style is abbrvnat
% The bibliography always appears last, after the appendices.

\bibliographystyle{abbrvnat}

\bibliography{l4proj}

\end{document}
