% REMEMBER: You must not plagiarise anything in your report. Be extremely careful.

\documentclass{l4proj}

    
%
% put any additional packages here
%

\begin{document}

%==============================================================================
%% METADATA
\title{Creating a server for protein disorder prediction based on deep neural networks}
\author{Ryan Gregson}
\date{March, 2023}

\maketitle

%==============================================================================
%% ABSTRACT
\begin{abstract}
    Every abstract follows a similar pattern. Motivate; set aims; describe work; explain results.
    \vskip 0.5em
    ``XYZ is bad. This project investigated ABC to determine if it was better. 
    ABC used XXX and YYY to implement ZZZ. This is particularly interesting as XXX and YYY have
    never been used together. It was found that  
    ABC was 20\% better than XYZ, though it caused rabies in half of subjects.''
\end{abstract}

%==============================================================================

% EDUCATION REUSE CONSENT FORM
% If you consent to your project being shown to future students for educational purposes
% then insert your name and the date below to  sign the education use form that appears in the front of the document. 
% You must explicitly give consent if you wish to do so.
% If you sign, your project may be included in the Hall of Fame if it scores particularly highly.
%
% Please note that you are under no obligation to sign 
% this declaration, but doing so would help future students.
%
\def\consentname {Ryan Gregson} % your full name
\def\consentdate {1 March 2023} % the date you agree
%
\educationalconsent


%==============================================================================
\tableofcontents

%==============================================================================
%% Notes on formatting
%==============================================================================
% The first page, abstract and table of contents are numbered using Roman numerals and are not
% included in the page count. 
%
% From now on pages are numbered
% using Arabic numerals. Therefore, immediately after the first call to \chapter we need the call
% \pagenumbering{arabic} and this should be called once only in the document. 
%
% Do not alter the bibliography style.
%
% The first Chapter should then be on page 1. You are allowed 40 pages for a 40 credit project and 30 pages for a 
% 20 credit report. This includes everything numbered in Arabic numerals (excluding front matter) up
% to but excluding the appendices and bibliography.
%
% You must not alter text size (it is currently 10pt) or alter margins or spacing.
%
%
%==================================================================================================================================
%
% IMPORTANT
% The chapter headings here are **suggestions**. You don't have to follow this model if
% it doesn't fit your project. Every project should have an introduction and conclusion,
% however. 
%
%==================================================================================================================================
\chapter{Introduction}

% reset page numbering. Don't remove this!
\pagenumbering{arabic} 

\section{Motivation}

Disordered proteins cause neurodegenerative diseases such as Alzheimer’s, Parkinson’s and Huntington’s. These intrinsically disordered proteins do not have a fixed or ordered three-dimensional structure and there are significant differences between the amino acid sequences of intrinsically disordered proteins (IDPs) / intrinsically disordered regions (IDRs) compared to structured globular (functional) proteins \citep{disordered_proteins_diseases_paper}. The structural order of these sequences can be manually labelled from experiments with x-ray crystallographic analysis \citep{idp_wiki}. These experiments can be difficult and the number of newly occurring proteins submitted to the Protein Data Bank (PDB) \citep{pdb} each year makes it difficult to manually experiment with each one, meaning many proteins’ disorder structure have not been assessed. We can use bioinformatic tools to label the disordered regions in these protein sequences by creating a model that can learn the differences between ordered and disordered amino acid sequences. With more labelled sequences, doctors and researchers can understand more about disordered protein sequences. A better understanding of these regions helps research into the development of drugs to counteract diseases and contributes to keeping a well-maintained knowledge database of proteins.  

Deep learning approaches have become incredibly popular recently, and these approaches currently dominate classification tasks. Deep learning architectures are prevalent in classifying the IDRs in protein sequences, where state of the art predictors such as SPOT-DISORDER2 \citep{spot_disorder_2} also have their own online protein disorder prediction servers for others to use their deep learning model.

\section{General problem and aim}

Protein sequences are made up of amino acids. The problem we are attempting to solve is to see if we can use these sequences to predict whether each amino acid is ordered or disordered, hence identifying the intrinsically disordered regions. We will produce deep neural networks which will take in protein sequences and predict the disordered regions within them. 

The aim of this project is to explore how different deep learning approaches and architectures perform at predicting protein disorder. To assess these deep learning methods, we will: 
\begin{itemize}
    \item Discuss current solutions to protein disorder prediction.
    \item Discuss our approach for creating a protein disorder classifier.
    \item Design and implement different deep neural networks, such as a convolutional neural network (CNN) and recurrent neural network (RNN). 
    \item Evaluate our implemented neural networks using standard metrics and the test datasets proposed by CASP \citep{casp}, which have been used for evaluating protein disorder prediction models. 
\end{itemize}
An aim alongside comparing the deep neural networks is to create a web server, where a user can input a protein sequence via a web page form and see the predictions of disordered regions. This will make my models accessible, which will result in individuals being able to interact with my developed solution easily. 


%==================================================================================================================================
\chapter{Background}
\label{chap:background}

\section{Biology overview}

\subsection{What is a protein?}

\begin{figure}
    \centering
    \includegraphics[width=\linewidth]{images/bg_seq.pdf}

    \caption{Transmembrane protein 218 amino acid sequence as presented by UniProt \citep{prot_A2RU14}.}

    \label{fig:sequence example} 
\end{figure}

Proteins are a linear sequence of amino acids \citep{prot_struct_lib}. An example of this sequence is the protein, Transmembrane protein 218, with a linear sequence of 115 amino acids \ref{fig:sequence example} \citep{prot_A2RU14}. There are twenty possible amino acids \citep{aa_wiki} protein sequences can be built from, and each of these amino acids have a unique side chain which have different structural properties. Different proteins have different sequences of amino acids. The amino acids in the sequence will "interact with each other to form a well-defined, folded protein" \citep{protein_folding}. The folded protein with have a three-dimensional structure, and this three-dimensional structure determines the protein's function \citep{bbc_bitesize}. This function can be supporting frameworks inside cells, catalysing biological reactions, communication between different parts of the body and proteins are used for function inside the immune system \citep{bbc_bitesize}. Therefore, it is important that the folding of proteins is done correctly as misfolded proteins will lose their structure and function, which can lead to diseases. 

\subsection{Intrinsically disordered proteins (IDPs)}
\label{chap:background sec:IDPs}

IDPs are different to structured proteins. These unstructured, disordered proteins are suggested to cause a number of diseases. Their amino acid sequences differ and contain a “low content of bulky hydrophobic amino acids and a high proportion of polar and charged amino acids” \citep{idp_wiki}. In comparison to structured proteins that fold correctly these sequences cannot fold into stable globular proteins, meaning that the protein lacks an ordered three-dimensional structure. The root cause of this lack of structure, causing disorder, is the proteins’ primary structure: the amino acid sequence. When disordered amino acids are grouped, this is labelled as an intrinsically disordered region (IDR) and is where the misfolding occurs. Therefore, making predictions about the probability that an amino acid residue is ordered or disordered within a protein sequence will help determine where this incorrect folding will occur. This means by focusing on the amino acids’ structure we can determine the outcome of the protein functioning correctly. 

A current problem is the lack of labelled data about the structure of protein sequences as determining these via experiments such as nuclear magnetic resonance (NMR) and small-angle X-ray scattering (SAXS) is very costly and time-consuming \citep{disordered_prot_genus_camelus}. To tackle this challenge, bioinformatic tools have been created for protein sequence structure labelling. These tools are often accessible via a web server, such as the DISOPRED server \citep{disopred3_paper}, which uses machine learning methods to make predictions. This project will produce a similar tool for classifying intrinsically disordered proteins after experimentation with different deep learning architectures and approaches.

\section{Deep learning overview}

\subsection{Deep Neural Networks}

Deep neural networks are made up of stacked neural network layers. As these layers are stacked, the output of one layer will be used as input to the next layer in the network. Each of these layers performs feature extraction on their input. Each layer can consist of an arbitrary number of neurons, each with their own parameter weight which is learnt during training. During training, inputs are classified by the model during a forward pass through the model. These predictions are compared to the true value using a loss function. Through the process of backpropagation \citep{Goodfellow-et-al-2016}, the outcome of the loss function is used and the parameter weights for each neuron within the network are changed in an aim to improve the models' predictions. To complete this training step, an optimisation method will update each layer of the network, changing the weights of the model as proposed by backpropagation. This training procedure is done with each item in the training set and for multiple iterations of the entire training set, to solve the optimisation problem by minimising the loss and finding optimum weights for each neuron. The machine learning framework, PyTorch \citep{pytorch}, provides the functionality to build neural network layers that extract features; use a loss function with an implemented backpropagation method, and create an optimiser to update a deep neural network model during training. PyTorch also provides efficient tensor computations and can utilise GPUs while training. This software will be useful for creating our deep neural networks due to it's efficiency and comprehensive components.

\subsection{Convolutional Neural Networks (CNNs)}

A CNN is a type of neural network that uses convolution kernels to extract features from an input. Convolution kernels can be 1D kernels of size $1\times K$ primarily used for sequence or series data, or 2D kernels of size $N\times M$ primarily used for classifying images. These kernels move across a grid shaped input, checking if a feature is present. With our protein sequences, we can use a kernel of $1\times K$ to slide along a $1\times sequence length$ grid, treating each amino acid as a different input channel and capturing unique. We can also represent protein sequences as images by creating a $20\times sequence length$ length grid, where each amino acid is a $20\times 1$ vector and this lets us use a different 2D convolution kernel over the protein sequence. These are discussed further in our design (\ref{chap:design}). 

In classification tasks convolution kernels are translation-invariant and require few parameters. These are beneficial as this means it does not matter where the location of the disordered region exists in the sequence. Less parameters means the model can be trained with smaller training sets, which is good as the labelled disorder datasets from Disprot \citep{disprot} is small in comparison to other deep learning tasks. Papers such as DISOPRED3 have stated further annotated training labels will be beneficial to prediction models \citep{disopred3_paper}; therefore, using a model that can cope with smaller datasets should be effective.

During most classification tasks, input of the same shape is passed to the CNN. However, with our protein sequence data the lengths of our sequences range from 19 \citep{prot_P0C8E0} to 34350 \citep{prot_Q8WZ42}, so we would need to unnecessarily pad many sequences with lots of redundant blank values. Therefore, we have taken an approach to use a fully convolutional network (FCN). Modern classification networks have been adapted to fully convolutional networks by Shelhamer and Long \citep{fcn_seg}, demonstrating that classification can be done using inputs of arbitrary size. Thus, for these protein sequences of varying length, an FCN will be suitable at handling these inputs. 

\subsection{Recurrent Neural Networks (RNNs)}

\section{Review of current approaches}
\label{sec:current approaches}

There are currently many approaches to protein disorder prediction, with over 100 different predictors proposed \citep{Zhao:22}. Disorder classifiers implement different methods with the main approaches regarded as: composition-based methods (PONDR VLXT \citep{Romero:01}); hydropath-based methods (DISOPRED \citep{Ward:04}); structural methods (IUPred \citep{Dosztanyi:18}); physiochemical methods (FoldIndex \citep{Prilusky:05}); machine learning methods (DISOPRED3 \citep{disopred3_paper}) and the use of meta-predictors (Metapredict \citep{Emenecker:21}). The Critical Assessment of Structure Prediction (CASP) \citep{casp} motivated solutions for identifying disordered regions in target proteins between 2010-12 using CASP9 and CASP10. Results from the CASP10 assessment \citep{CASP10} show that the top disordered region classifiers belong to machine learning and meta-predictor methods \citep{Zhao:22}. CASP has discontinued the evaluation of IDR prediction, however the Critical Assessment of Intrinsic Protein Disorder community \citep{Necci:21}, continued to release withheld testing datasets to assess new prediction tools. These assessments found that deep learning predictors had the best performance at identifying disordered regions accurately \citep{Zhao:22}. This motivates a reason for approaching this task using a deep learning method.

As well as performing well for intrinsic disorder prediction deep learning methods also dominate in other protein structure prediction tasks such as AlphaFold \citep{Jumper:21}. Improved technology has assisted these neural networks to feasibly increase their depth as well as the increase of digitalised labelled data. Studies have analysed different deep learning architectures for protein disorder prediction and state: “that the architectures of current deep learners are considerably diverse,” and there is no decided optimal architecture yet \citep{Zhao:22}. This motivates us to experiment and compare different deep learning architectures. Furthermore, the top performing models in CAID “utilize deep convolutional and/or recurrent neural networks,” \citep{Hu:21} demonstrating these architectures are important for classifiers tackling this prediction task.

\subsection{Convolutional neural networks}

A model using a convolutional neural network architecture will process the amino acid sequence in a sliding window fashion using a convolution kernel. Older competitive machine learning based methods such as DISOPRED, FoldIndex and IUPred use a sliding window within their algorithm, which justifies the feasibility of a convolution kernel. There are not many prominent models which solely use convolutional layers for protein disorder classification. However, two studies propose using a CNN for similar tasks: predicting DNA-protein binding \citep{Zeng:16}, and protein secondary structure prediction \citep{Ema:22}. The first study concludes that experimentation of different CNN architecture design demonstrated improvements to a state-of-the-art solution and this work “will be important for sequence-based tasks in genomics,” \citep{Zeng:16}. This suggests experimentation using a CNN will be useful for classification of disordered regions as we will be using a sequence-based approach with our protein sequences. The second study successfully implements a (2D) CNN with a hybrid machine learning layer and using protein sequence data for protein secondary structure prediction, which is a similar task to our protein disorder prediction problem. Although CNNs are very good at capturing local relationships, they are more limited in identifying medium and long-range sequence information. Wang et al proposes DeepCNF-D \citep{Wang:15}, which uses a deep convolutional neural fields architecture (comprising of a deep convolutional neural network and conditional random field) to reach state-of-the-art performance on the CASP benchmark datasets. The result of using a CNF architecture demonstrates the benefit of capturing complex dependencies from the amino acid sequences.

\subsection{Recurrent neural networks}

Findings that improved performance can result from capturing long-range and complex sequence dependencies that are encoded in intrinsically disordered proteins is useful. Recurrent neural networks can use their ability to maintain a hidden state to capture long-term dependencies, which is shown by the Long Short-Term Memory (LSTM) deep recurrent neural network SPOT-DISORDER \citep{Hanson:16}. This study achieved the best performance or matched best performing protein disorder prediction methods for all independent test sets and demonstrated “the capability of the LSTM technique to pick up long-range, non-local interactions” \citep{Hanson:16}. This approach also used a bidirectional recurrent neural network architecture, which is known to capture long-term dependencies and perform better than a traditional LSTM \citep{Graves:05}. Therefore, these bidirectional LSTM architectures are more likely to enhance protein disorder classification methods and should be considered with our approach.

\subsection{Ensemble approach}

As no definitive optimal architecture has been found, taking the consensus from multiple predictions using different models with different architectures can improve prediction accuracy. An improvement to the previous LSTM version of SPOT-DISORDER takes this ensemble approach, supporting that “the use of an ensemble predictor minimizes the effect of generalization errors between models” \citep{spot_disorder_2}. This SPOT-DISORDER2 ensemble uses CNN and RNN (LSTM) models, which improves prediction performance by removing “spurious false predictions” \citep{spot_disorder_2}. This paper finds benefits of using multiple models and highlights the benefit of experimenting with different architectures. These ensemble methods have also been motivated by meta-predictors which use other disorder predictors to reach a consensus prediction \citep{Emenecker:21}. These meta-approaches succeed at reaching top results and demonstrate the need for exploration of different model architectures.

\subsection{Accessibility of disorder classifiers}

Having multiple prediction models also allows researchers to create their own consensus of disorder prediction using different models. It is important that these prediction tools are easily accessible. Most trained classifiers are accessible to download and use with protein sequence data, which is beneficial for bioinformaticians. Furthermore, many studies produce an online web server to benefit less technical researchers, where they can make predictions on protein sequences using the trained classifiers via their browser. With this easy access, research can benefit from accurate predictions about protein sequences. These web servers display highlighted disordered regions within the protein sequence or graphs demonstrating the probability that a region is intrinsically disordered. This makes it clear from visual inspection where the protein is likely to misfold.

\subsection{Summary}

The findings of these current approaches demonstrate that intrinsic disorder prediction in proteins is successful with deep learning classifiers and that the exploration of different accurate models will allow for future ensemble and meta predictions to be taken. The use of protein disorder prediction by researchers also makes it clear why an easily accessible web server is important in allowing interaction with the prediction tool.


%==================================================================================================================================
\chapter{Analysis}

\section{General deep learning problem}

In our discussion of intrinsically disordered proteins (\ref{chap:background sec:IDPs}), we demonstrate that the sequences of these proteins are different compared to ordered protein sequences. If we evaluate which amino acids appear within specific regions and where they appear, we can draw conclusions about where these disordered regions are. Therefore, we will use a supervised learning approach with our deep neural network, like current approaches, by using these amino acid sequences alongside their labelled disordered regions.  

\section{Our dataset}

As the identification of these disordered regions requires complex experimentation (\ref{chap:background}), we must use a curated database of disordered proteins. The Disprot database \citep{disprot} has been used for a variety of machine learning tasks such as the training and testing of DISOPRED3 \citep{disopred3_paper}, PONDR \citep{Xue:10}, MFDp2 \citep{Mizianty:13} and SPOT-DISORDER \citep{Hanson:16}, which all involved classifying disordered proteins. This database is regarded as the “major repository of manually curated data for intrinsically disordered proteins” \citep{Quaglia:22} and is widely used by researchers studying disordered proteins. Disprot stores information about the positions of IDRs in sequences and uses a UniProtKB accession number to identify the protein. This identifier will allow us to retrieve the full protein sequence from the UniProt database \citep{uniprot:22}, another manually curated database for proteins, to help us train our network. 

\section{Labelled data}

\begin{figure}
    \centering
    \includegraphics{images/groundtruth.pdf}    

    \caption{The Osteocalcin protein \citep{prot_Q8HYY9} and it's disorder label. This ground truth label, represents disordered amino acids with a D and ordered amino acids are represented with O. Here we see the disordered region is at the start of this small protein sequence}

    \label{fig:groundtruth} 
\end{figure}

Compared to other problems solved by deep neural networks, such as protein structure prediction \citep{Jumper:21}, classifying images \citep{Xie:17} and language modelling \citep{Devlin:18}, the size of our manually labelled dataset is much smaller. Disprot only contains around 2500 protein sequences. However, we are not classifying an entire sequence as either ordered or disordered, we are classifying a single amino acid within the sequence as ordered or disordered as seen in Figure \ref{fig:groundtruth}. This is because we know the region of the protein can be disordered due to the characteristics of the grouped amino acids. This gives us sufficient labelled data. Using these labelled sequences as our ground truth, we can utilise our deep neural networks to recognise patterns and relationships about the amino acid sequence that causes intrinsic disorder in proteins through our supervised learning approach.

\section{Approaches}

Machine learning algorithms cannot operate on our protein sequences directly. Therefore, to ensure our model is equipped to handle the various amino acids, we will consider them as distinct categories of data and employ feature encoding techniques to generate a suitable numerical input \citep{Kumar:20}.

\subsection{One-hot}

Our first feature encoding approach is one-hot encoding \citep{One-hot:wiki} \citep{Fawcett:21}. We can treat each amino acid as a vector by using the one-hot encoding technique. Our vectors will have a size of 20 because of the twenty possible amino acids, therefore as these feature vectors are not excessively large, this one-hot approach is suitable. This has let us transform a raw amino acid sequence into a feature matrix, of shape $20 \times sequence length$, which now represents the sequence.

\subsection{Position-Specific Scoring Matrix (PSSM)}

Our other feature encoding approach is to generate a position-specific scoring matrix (PSSM) \citep{PSSM:wiki}. This technique is used in many protein structure problems, such as predicting protein secondary structure \citep{McGuffin:00}, identifying sequence-contact relationships \citep{Wang:17} and used within current approaches previously discussed in our review of the literature. A PSSM is a “matrix that involves information about the probability of amino acids or nucleotides occurrence in each position, which is derived from a multiple sequence alignment” \citep{Mohammadi:22}. These PSSMs provide an informative representation about the amino acids at positions within the protein sequences which will be helpful for our network to identify patterns and relationships within protein sequences. Furthermore, as PSSMs are also of shape $20 \times sequence length$, the neural network models will be able to take both one-hot encoded sequence matrices and PSSMs as input without any changes, allowing us to compare these different feature encoding methods.

\section{Architectures}

\section{Evaluation metrics}
\label{chap:analysis sec:evaluation}
With our trained deep neural network, we will evaluate its predictions using similar approaches to the literature. As this is a binary classification task classifying amino acids as either ordered or disordered we can define our outcomes as: 

\begin{itemize}    
    \item True positive (TP): When the model correctly predicts that a disordered residue is disordered.
    \item True negative (TN): When the model correctly predicts that an ordered residue is ordered.
    \item False positive (FP): When the model incorrectly predicts that an ordered residue is disordered.
    \item False negative (FN): When the model incorrectly predicts that a disordered residue is ordered. \citep{google_TP}
\end{itemize}

By computing these outcomes, we can evaluate the accuracy, precision, recall (sensitivity) and specificity of our model. These metrics are widely used in various machine learning tasks as standard performance measures \citep{BC:wiki}. We can also calculate the F1-score which is the harmonic mean of the precision and recall and is another useful measure of how accurate our model is. Furthermore, as our dataset has a bias of ordered labels, we will consider the Matthews correlation coefficient (MCC), which is a more reliable measure when dealing with an imbalanced or disproportionate dataset \citep{Chicco:20}.

With these evaluation metrics we will be able to compare our approaches and architectures against each other. We can also compare our models to known current approaches using the CASP datasets which have been used for benchmarking these protein disorder prediction models \citep{casp}.

\section{Django server}

In addition to evaluating deep learning models, it is necessary to create a Django-based server that enables users to submit a protein sequence and obtain a comprehensible representation of the disordered regions of the protein. From the literature, papers proposing protein disorder prediction tools also often set up an online server so their tool can be used such as UCL's PSIPRED workbench \citep{DISOPRED:server} and the recently published Metapredict server \citep{Metapredict:server}. These protein prediction servers are important as they allow researchers to easily access these tools, allowing them to understand the structure and function of proteins which is useful for protein engineering \citep{prot_engineering:wiki}. 

A brief analysis of the server’s requirements follows following the MoSCoW prioritisation method \citep{moscow}: 
\subsection{Functional requirements}

\begin{itemize}    
    \item Must have: The server will have a web interface, so it is easily accessible for use by researchers.
    \item Must have: Users can submit a single protein sequence via a form.
    \item Must have: The server can make a prediction on a protein sequence using a deep neural network. This model will be chosen from our evaluation.
    \item Should have: The web interface will have a clear visualisation displaying the sequence to the user with clearly identified IDRs.
    \item Could have: The web interface will have a graphical representation of the identified IDRs.
    \item Will not have this time: The visualisation of the disordered sequence cannot be saved.
    \item Will not have this time: The server will not allow multiple sequences to be entered in a single form submission.
\end{itemize}

\subsection{Non-functional requirements of our server}

\begin{itemize}    
    \item Must have: The web interface will be intuitive and easy to use. Both when entering information in via the form and interpreting the outputted sequence with highlighted disordered regions.
    \item Should have: The web application should be containerised to follow best practices and be easily portable.
    \item Could have: The Django server could interact with an external database for efficient lookup of previously entered sequences and their representations.
    \item Will not have this time: The web page will not be hosted and accessible via a searchable domain by the conclusion of the project.
\end{itemize}


\newpage
What is the problem that you want to solve, and how did you arrive at it?
\section{Guidance}
Make it clear how you derived the constrained form of your problem via a clear and logical process. 

%==================================================================================================================================
\chapter{Design}
\label{chap:design}

\section{Data processing}

We will need to build a data processing pipeline where we will gather specific information about each unique protein from the DisProt dataset, such as their full sequence and the location of all the disordered regions, so we can create our ground truth label. This ground truth label is vital for our supervised learning task. Given our DisProt dataset, we identify unique proteins using a UniProt accession identifier. Therefore, we should use this accession identifier as a key in hash map data structures. This will let us easily lookup the feature encoded sequence for our DNN's input and a sequence’s true disordered regions which let us create our label for evaluating our prediction. We build the label by creating a vector the size of the sequence, and from our looked up disordered regions of a given protein we will set the amino acids where intrinsic disorder occurs to 1 (as disorder is our positive class) and ordered amino acid positions will be set to 0. This vector will be used in our loss function which is discussed later \ref{sec:loss design}. 

Processing is also required to encode these sequences before they are looked up. We have two different vectorised representations of these sequences: a one-hot encoded matrix and a PSSM. The one-hot encoded matrix is a dense matrix and is made using standard one-hot feature encoding. An example of the beginning of a one-hot encoded sequence is shown in Figure \ref{fig:feature encoding}. The PSSM can be generated with a bit more involved work using an alignment search tool first. This tool alongside a protein database can generate our PSSMs. This processing gives us information about the conservation of amino acids between similar sequences which conveys more information about the sequence than a one-hot representation. An example of a PSSM is shown in Figure \ref{fig:feature encoding}, and each PSSM can be parsed into a 2D data structure which creates the appropriate matrix format our deep neural network can use. A limitation of using PSSMs is that they can be much more computationally expensive to create than our one-hot encoding approach as they perform sequence alignment using a very large protein database such as UniProt \citep{uniprot:22}. For this reason, we have also implemented a one-hot encoding approach to see if generating PSSMs is worth the computational expense. 

\subsection{Training, validation and testing data}

Our final key processing step is to divide our dataset into a training, validation and test set before any learning takes place. We follow the common approach of using 20\% withheld data for the test set, and the rest will be used for training and validating the model. We will further separate this training and validation set by 75\% and 25\% respectively so that the overall split of training, validation and testing data is 60\%, 20\% and 20\% respectively. Random sampling (with a seed to ensure reproducible splitting) will be done to separate these sequences. However, as protein sequences can have a similar ancestor, this means they can be homologous and there may be data leakage between datasets. This means information from the test dataset can accidentally appear in our training dataset \citep{Cook:22}. To detect this data leakage, we will look for homologous sequences by employing a homology clustering search. We will avoid any data leakage by moving sequences between datasets if they are homologous, such that all datasets will not have homologous sequences between them. Despite movement of sequences, we still aim to have a division of approximately 60\%, 20\%, 20\% between our three datasets and will verify this before we perform further evaluation. 

\begin{figure}
    \centering
    \begin{subfigure}[h]{0.45\textwidth}
        \centering
        \caption{One-hot encoding}
        \label{fig:feat1}
        \begin{tabular}{|c|cccccc|}
        \hline
          & M & A & S & R & E & ...\\ \hline
        A & 0 & 1 & 0 & 0 & 0 & \\ 
        C & 0 & 0 & 0 & 0 & 0 & \\ 
        D & 0 & 0 & 0 & 0 & 0 & \\ 
        E & 0 & 0 & 0 & 0 & 1 & \\ 
        F & 0 & 0 & 0 & 0 & 0 & \\ 
        G & 0 & 0 & 0 & 0 & 0 & \\ 
        H & 0 & 0 & 0 & 0 & 0 & \\ 
        I & 0 & 0 & 0 & 0 & 0 & \\ 
        K & 0 & 0 & 0 & 0 & 0 & \\ 
        L & 0 & 0 & 0 & 0 & 0 & ...\\ 
        M & 1 & 0 & 0 & 0 & 0 & \\ 
        N & 0 & 0 & 0 & 0 & 0 & \\ 
        P & 0 & 0 & 0 & 0 & 0 & \\ 
        Q & 0 & 0 & 0 & 0 & 0 & \\ 
        R & 0 & 0 & 0 & 1 & 0 & \\ 
        S & 0 & 0 & 1 & 0 & 0 & \\ 
        T & 0 & 0 & 0 & 0 & 0 & \\ 
        V & 0 & 0 & 0 & 0 & 0 & \\ 
        W & 0 & 0 & 0 & 0 & 0 & \\ 
        Y & 0 & 0 & 0 & 0 & 0 & \\ \hline
        \end{tabular}
    \end{subfigure}
    \begin{subfigure}[h]{0.45\textwidth}
        \centering
        \caption{Position-specific scoring matrix}
        \label{fig:feat2}
        \begin{tabular}{|c|cccccc|}
        \hline
        & M & A & S & R & E & ...\\ \hline
        A & $-2$ & $\phantom{-}4$ & $\phantom{-}0$ & $-2$ & $-1$ & \\
        C & $\phantom{-}0$ & $\phantom{-}0$ & $\phantom{-}0$ & $-2$ & $-2$ & \\
        D & $-3$ & $-2$ & $\phantom{-}0$ & $-1$ & $\phantom{-}0$ & \\
        E & $-4$ & $-3$ & $-2$ & $-1$ & $\phantom{-}4$ & \\
        F & $\phantom{-}1$ & $-1$ & $-1$ & $-2$ & $-2$ & \\
        G & $-3$ & $\phantom{-}0$ & $-1$ & $-2$ & $-2$ & \\
        H & $-2$ & $-2$ & $-2$ & $-1$ & $-1$ & \\
        I & $\phantom{-}2$ & $\phantom{-}0$ & $-1$ & $-2$ & $-2$ & \\
        K & $-2$ & $-2$ & $-1$ & $\phantom{-}0$ & $\phantom{-}0$ & \\
        L & $\phantom{-}3$ & $\phantom{-}0$ & $-1$ & $-1$ & $-2$ & ... \\
        M & $\phantom{-}7$ & $-1$ & $-1$ & $-1$ & $-2$ & \\
        N & $-3$ & $-2$ & $\phantom{-}0$ & $-1$ & $-1$ & \\
        P & $-3$ & $-1$ & $-1$ & $-2$ & $-2$ & \\
        Q & $-2$ & $-2$ & $-2$ & $\phantom{-}2$ & $\phantom{-}0$ & \\
        R & $-3$ & $-4$ & $-3$ & $\phantom{-}3$ & $-2$ & \\
        S & $-3$ & $\phantom{-}0$ & $\phantom{-}4$ & $-2$ & $\phantom{-}0$ & \\
        T & $-1$ & $-1$ & $\phantom{-}0$ & $-2$ & $-1$ & \\
        V & $\phantom{-}1$ & $\phantom{-}0$ & $-1$ & $-2$ & $-1$ & \\
        W & $\phantom{-}0$ & $-1$ & $-1$ & $-1$ & $-2$ & \\
        Y & $\phantom{-}0$ & $-1$ & $-1$ & $-1$ & $-1$ & \\ \hline
        \end{tabular}
    \end{subfigure}
    \caption{Two possible ways of encoding our protein sequences. \subref{fig:feat1} shows a one-hot encoding approach, where 1.0 identifies the amino acid at each sequence position. \subref{fig:feat2} shows the PSSM, where larger positive values are given to amino acids that are expected to appear at each position of the sequence. This protein is Human adenovirus C serotype 5 (HAdV-5) (Human adenovirus 5), the first protein sequence in the DisProt dataset, and we show the first five amino acids from this sequence.
    }\label{fig:feature encoding}
\end{figure}

\section{Our neural networks and how we train them}

\subsection{Model input}

Our first the step in training a DNN is giving the model input. The input to our models will be our feature encoded sequence (either a one-hot or position-specific scoring matrix), which will have shape $20\times sequence length$. We will only give a model one protein sequence to evaluate at each forward pass as we know that sequence lengths differ between proteins; therefore, we cannot batch multiple sequences together. Batching multiple sequences would create an unexpected, irregular shape, which neural network layers cannot handle. Thus, we use a batch size of one as input to our DNN models.

\subsection{CNNs}

Our CNNs must actually be fully convolutional neural networks (FCNs) because our protein sequences have different lengths, therefore we cannot use a fully connected layer as it expects a fixed size tensor \citep{fcn_seg}. So, we will only use convolution layers which create feature maps for our model to identify information about the protein sequence and used for the disorder classification task. These layers must also reshape our input to a $1\times sequence length$ length tensor for our loss function.

There are two different convolutional neural networks we can design: one that treats our protein sequence as a black and white image and uses 2D convolutions, or one that treats each row in the encoded matrix (different amino acids) as a different feature representation of the input sequence and uses 1D convolutions over these separate feature input channels. 

Usually, a 2D convolutional neural network is applied to images, therefore given our $20\times sequence length$, the height of the proposed image is the number of possible amino acids, and the width of the image is determined by the sequence. Our input to a 2D convolutional neural network must follow the [batch size, height, width, channels] constraint, therefore our input will be a tensor of shape [1, 20, sequence length, 1] due to batching of size 1 and treating the colour channel as black and white. 

As discussed earlier, we want our convolution kernel to behave like a sliding window, therefore it must overlay the entire height of the image to take in a full context window of amino acid information. Capturing the entire height involves using a kernel of size 20, however we cannot appropriately pad the width of the sequence with this kernel shape such that the same width is returned after convolving with a kernel of this shape. The width cannot be changed as it is necessary to compare each amino acid disorder classification against its true label. Therefore, we will use a non-square convolution kernel so that the width of our kernel is an odd value and appropriate padding can be included. 

After the first convolution layer, layers after this will be dependent on the feature map created by this first layer and further layers. The number of feature channels produced by this first convolution layer can be experimented with and we have chosen to use ten hidden channels. After further convolution layers we want our final feature map output to take shape $1\times sequence length$ so our predictions can be compared against the true label. 

Our other fully convolutional network uses 1D convolutions. Usually, a 1D convolutional neural network is applied to sequences and accepts an input of $1\times sequence length$ shape. For our neural network, we will treat each amino acid feature representation as a separate input channel, giving us 20 channels, each with shape $1\times sequence length$. The input to the model must follow the [batch size, width, input channels] constraint, therefore our input will be a tensor of shape [1, sequence length, 20]. This 1D convolution kernel also behaves like a sliding window, and a kernel will produce one output feature map which represents all the amino acid input channels. This ensures all amino acid information is captured for a context window at each position along the sequence. Furthermore, an odd numbered kernel must be chosen so that appropriate padding can be included to the width of our sequence similarly to the 2D CNN. Like the 2D CNN, all further convolution layers can be changed within the network as long as the final layer produces a $1\times sequence length$ feature map as its output for comparison with our ground truth labels.

\subsection{RNN}

Our RNNs will use the long short-term memory (LSTM) architecture. These LSTMs combat the vanilla RNN’s vanishing gradient problem, and they can identify long-term dependencies which we know is useful from our literature analysis. Usually, LSTMs are applied to sequential data, and they expect a batch size, a series of timesteps and a number of feature channels. These features are represented the in same way as our 1D CNN initial feature channels; where each amino acid row from our vectorised protein sequence is used to represent the sequence as a separate feature vector. We can also treat each amino acid in the sequence as a timestep. Lastly, as using batching requires our sequences to be the same length, which we do not have, we will continue to batch with a single sequence in each batch instead of padding our sequences. Therefore, our input to the LSTM model will be a tensor of shape [1, sequence length, 20]. 

We also need to initialise our hidden state for our LSTMs. These will be initialised with all zeros every forward pass, so they are reset each sequence. It is common practice to set the hidden state to an all zero tensor with a shape considering the batch size and the number of hidden dimensions we want. In our model we will choose to have 10 hidden dimensions. 

Finally, our LSTM will be a bidirectional LSTM (BLSTM). This performs better than a traditional LSTM and can capture long-term dependencies about the IDPs using information from the start and end of these sequences. With information flowing in both directions, we can capture more contextual information about each amino acid. Furthermore, a protein sequence does not have a flow of information like a time series, and the start (N-terminus) and end (C-terminus) of a sequence is chosen arbitrarily, meaning this protein could be written in the opposite direction. Thus, it is useful to assess each sequence from both directions. A limitation of implementing a bidirectional network is that it will be much slower to compute because forward propagation requires both forward and backward recursions, which creates a long dependency chain and affects our gradient calculation using backpropagation \citep{Zhang:2021}.  

Like our other networks, our LSTM will use a final linear layer to reduce our feature map down to a single feature (the disorder classification), so that a $1\times sequence length$ output is returned for the rest of our training loop to handle.

\subsection{Activation functions}

Between each layer in our neural networks, we use the rectified linear unit activation function (ReLU) as it is “the most widely used and a goto activation function for most types of problem,” \citep{Keerthi:22}. This is because ReLU can be more computationally efficient than other activation functions, and using ReLU in networks tends to give a better convergence performance \citep{BerenLuthien:16}. After our final layer, we employ the sigmoid activation function because this gives us a value in the range [0, 1]. We can treat this value as a probability as it shows how likely that this position in the sequence is a disordered amino acid. This value alongside a threshold of 0.5 will be used to identify the predicted IDRs.

\section{Discussion about loss function and optimisation}
\label{sec:loss design}
To allow our deep neural network to learn and update its neurons weights we must use an optimisation method and an appropriate cost function. We know our problem is a binary classification task from our analysis of the problem; therefore, we can use binary cross-entropy (also known as log loss) as our loss function \ref{fig:bceloss} \citep{Godoy:18}.  

With these two possible outcomes we are treating a disordered amino acid as the positive class and an ordered amino acid as the negative class. From analysing our DisProt dataset, we see that our two classes are imbalanced: there are a lot more ordered amino acids within these intrinsically disordered proteins, specifically there are approximately five ordered amino acids for every disordered amino acid. One way of handling this imbalanced dataset is using a weighted loss function. For this, we calculate the number of ordered and disordered amino acids in our training dataset. Then, we can calculate a weight multiplier to be used to give the disordered amino acids loss more weighting. By scaling the loss of predictions about the amino acids we will emphasise predictions that should take a disordered label which will help the model better identify this disproportionate class because the model will be penalized more heavily for misclassifying disorder. 

This binary cross-entropy loss function is useful for gradient-based optimisation algorithms because it is easy to compute and is sensitive to small changes in the predicted probabilities. We plan to use stochastic gradient descent (SGD), which is a popular machine learning optimisation algorithm. We will also experiment using the Adam optimiser, however as this optimisation method usually works well with large datasets and SGD is known to generalise better than Adam \citep{Zhou:20}, I believe SGD will be a better optimisation method and produce a better model. 

\begin{figure}[h]
    \centering
    \begin{equation}
    \ell(y, \hat{y}) = -\frac{1}{N} \sum_{i=1}^{N} w_i [ y_i \log(\hat{y}_i) + (1-y_i)\log(1-\hat{y}_i)]
    \end{equation}
    \caption{Binary Cross-Entropy (Log Loss) Function. This equation represents the loss function used in binary classification tasks. For our task, $y$ is the true label of order or disorder, $\hat{y}$ is the predicted probability (i.e., the output of the sigmoid function), and $N$ is the number of amino acids in the sequence. The function measures the difference between the predicted and true labels. The weight vector $w$, is used to penalise the model more when it is more certain about incorrect predictions of the disordered class. Having an all ones vector for $w$ is default when no weights are specified.}
    \label{fig:bceloss}
\end{figure}

\section{The design of our evaluation experiments}

During the training of our model, we will evaluate the performance of the optimisation task. We can use the total loss of our training set and validation set predictions to ensure our model is learning at a sufficient rate and is generalisable on withheld data. Using the validation dataset as our first withheld dataset, we can optimise hyperparameters, such as the learning rate and regularisation weights to make sure our validation loss decreases along with our training loss. Furthermore, we will plot loss curves to make it visually clear our model is generalisable to our validation dataset \citep{Brownlee:19}. With a generalisable model we are more likely to achieve correct classifications on future unseen data. 

After tuning the important hyperparameters for our different model configurations we will retrain our model using the 80\% of the DisProt dataset allocated to training data (this includes our training and validation datasets). Using the other 20\% of the DisProt dataset withheld for testing, we will evaluate our models using the evaluation metrics discussed in our analysis \ref{chap:analysis sec:evaluation}. Discussion about these metrics and the equations for these standard machine learning evaluation measures are given below. 

\begin{equation*}
Accuracy = \frac{TP + TN}{TP + TN + FP + FN} ,
\end{equation*}
where TP (true positives) is the number of correctly classified disordered amino acid instances, TN (true negatives) is the number of correctly classified ordered amino acids instances, FP (false positives) is the number of incorrectly classified disordered amino acids instances, and FN (false negatives) is the number of incorrectly classified ordered amino acids instances.

\begin{equation*}
Precision = \frac{TP}{TP + FP} ,
\end{equation*}
Precision measures the proportion of true positive predictions among all positive predictions.

\begin{equation*}
Recall = \frac{TP}{TP + FN} ,
\end{equation*}
Recall measures the proportion of true positive predictions among all actual positive instances.

\begin{equation*}
Specificity = \frac{TN}{TN + FP} ,
\end{equation*}
Specificity measures the proportion of true negative predictions among all actual negative instances.

\begin{equation*}
F1\ score = \frac{2 \times Precision \times Recall}{Precision + Recall} ,
\end{equation*}
F1 score is the harmonic mean of precision and recall and provides a balanced measure of both metrics.

\begin{equation*}
MCC = \frac{TP \times TN - FP \times FN}{\sqrt{(TP + FP)(TP + FN)(TN + FP)(TN + FN)}}
\end{equation*}
MCC measures the correlation between predicted and actual binary classifications, taking into account all four confusion matrix elements. It ranges from -1 to 1, with 1 indicating a perfect correlation, 0 indicating no correlation, and -1 indicating a perfect negative correlation.

\subsection{Evaluating on different datasets}
After our full evaluation is complete, we will finally train our model using all (100\%) of our data from the DisProt database as the separate datasets have served their purpose in helping us find the best model. Training with this extra data means that our model should have at least the performance accuracies as it did in our evaluation, and this is a best practice before deploying a machine learning model \citep{Brownlee:17}. After this step, our trained model is ready to be deployed on our web server and can be assessed using other unseen datasets. 

Furthermore, we will use the CASP10 dataset to evaluate our model. This dataset contains 94 protein sequences \citep{Moult:14} and has been used to assess many protein disorder classifiers. Details of other protein disorder prediction methods performances can be found in other papers; therefore, we can compare our findings against other approaches and architectures. 

\section{The web server}

\subsection{User interface}

The user interface was designed to be similar to other protein disorder prediction websites. All websites used an input form, stated what input was valid and displayed the output to the user in a timely manner for single sequences. From checking popular protein prediction web servers, I felt the most user-friendly websites included an example of what sequence input was valid, therefore I made this a consideration when building the website. 

We have two pages on our web server: the input form and the displayed disorder prediction.

\subsubsection{The input form web page.}

This page was made to be intuitive, with the main functionality is letting the user input a sequence. The input to this form can only be a valid protein sequence or FASTA file, which is explained on the web page. We have also included links to PDB, UniProt and DisProt to assist users in finding protein sequences. The wireframe for this page is shown in Figure \ref{fig:wireframe1}.

\subsubsection{The disorder prediction web page.}

This page displays the IDR predictions. We have used a sequence-based visualisation, which demonstrates which amino acids are disordered and where the disordered regions are. Other web servers such as ESpritz \citep{Walsh:11} have also used a sequence display, but this website is older and uses a ‘D’ or ‘O’ label next to the amino acid instead of a coloured label. We believe that the coloured approach is more effective at representing the disordered regions and has also been adopted by web servers that have been maintained and updated recently like PSIPRED \citep{DISOPRED:server}. The wireframe of this page is shown in Figure \ref{fig:wireframe2}.

\begin{figure}
    \centering
    \includegraphics[width=\linewidth]{dissertation/images/Index - form input.pdf}
    
    \caption{The home page of our web server. This wireframe shows that users can navigate to external protein database websites, and enter their protein sequence in FASTA format to our input form.}
    
    \label{fig:wireframe1} 
\end{figure}

\begin{figure}
    \centering
    \includegraphics[width=\linewidth]{dissertation/images/Index - disorder prediction.pdf}
    
    \caption{The disorder prediction web page. This wireframe demonstrates how our deep learning models classification of disordered regions will be displayed after a protein sequence has been submitted.}
    
    \label{fig:wireframe2} 
\end{figure}

\section{Tools and technologies}

The tools and technologies we used for our deep learning models and web server are mostly different, however the trained deep neural network model must be easily accessible on the web server. Using PyTorch for the deep learning and Django for the web server will allow us to integrate our trained PyTorch model into our web application easily using Python for consistency.

\subsection{Deep learning}

To create our different deep neural network architectures, we will use the PyTorch framework \citep{pytorch}. PyTorch lets us construct our models using their neural network library. Using Python also allows us to carry out our pre-processing and evaluation steps with different libraries that help us make efficient computations along with appealing and informative visualisations about the performance of our prediction task. Using Python also lets us utilise Google colab’s notebooks \citep{Bisong:19}, where we can use a GPU for decreased training times and include markdown throughout our notebook, which helps with the understanding of our code.

An external tool is required to generate the encoded features for our model's PSSM input. We have chosen to use MMSeqs2 (Many-against-Many sequence searching) \citep{Steinegger:17}. MMSeqs2 can perform multiple sequence alignment searches much faster than PSI-BLAST \citep{Altschul:97} and HHblits \citep{Remmert:12} which are other software suites used for PSSM generation. Despite HHblits giving slightly more accurate alignments, this task is much more computationally efficient with MMSeqs2, which still generates accurate alignments. Recent papers, such as NetSurfP-2.0 \citep{Klausen:19} have used MMSeqs2 to generate their PSSMs for large sequences due to MMSeqs2 being less resource intensive than HHblits and found little differences in the models' performance trained using PSSMs generated by the different suites.

\subsection{Web server}

To create our web server, we will use the Django web framework \citep{Django:05}. Web pages will be built using standard web technologies (HTML, CSS and JavaScript). In Django’s backend we will have the views which can process requests and manipulate the data as needed. If we save our trained neural network as a static file, then the views can load our model so that the web application can make use of it. Furthermore, this should be straightforward to do as Django is a Python web framework, PyTorch can be easily imported and loaded amongst the web application logic. 
Finally, we have used Docker \citep{Docker:14} along with Nginx \citep{Nginx:08} and Gunicorn \citep{gunicorn:wiki}, so that our web server can be containerised and is portable which is one of our initial requirements. 

%==================================================================================================================================
\chapter{Implementation}
What did you do to implement this idea, and what technical achievements did you make?
\section{Guidance}
You can't talk about everything. Cover the high level first, then cover important, relevant or impressive details.



\section{General points}

These points apply to the whole dissertation, not just this chapter.



\subsection{Figures}
\emph{Always} refer to figures included, like Figure \ref{fig:relu}, in the body of the text. Include full, explanatory captions and make sure the figures look good on the page.
You may include multiple figures in one float, as in Figure \ref{fig:synthetic}, using \texttt{subcaption}, which is enabled in the template.



% Figures are important. Use them well.
\begin{figure}
    \centering
    \includegraphics[width=0.5\linewidth]{images/relu.pdf}    

    \caption{In figure captions, explain what the reader is looking at: ``A schematic of the rectifying linear unit, where $a$ is the output amplitude,
    $d$ is a configurable dead-zone, and $Z_j$ is the input signal'', as well as why the reader is looking at this: 
    ``It is notable that there is no activation \emph{at all} below 0, which explains our initial results.'' 
    \textbf{Use vector image formats (.pdf) where possible}. Size figures appropriately, and do not make them over-large or too small to read.
    }

    % use the notation fig:name to cross reference a figure
    \label{fig:relu} 
\end{figure}


\begin{figure}
    \centering
    \begin{subfigure}[b]{0.45\textwidth}
        \includegraphics[width=\textwidth]{images/synthetic.png}
        \caption{Synthetic image, black on white.}
        \label{fig:syn1}
    \end{subfigure}
    ~ %add desired spacing between images, e. g. ~, \quad, \qquad, \hfill etc. 
      %(or a blank line to force the subfigure onto a new line)
    \begin{subfigure}[b]{0.45\textwidth}
        \includegraphics[width=\textwidth]{images/synthetic_2.png}
        \caption{Synthetic image, white on black.}
        \label{fig:syn2}
    \end{subfigure}
    ~ %add desired spacing between images, e. g. ~, \quad, \qquad, \hfill etc. 
    %(or a blank line to force the subfigure onto a new line)    
    \caption{Synthetic test images for edge detection algorithms. \subref{fig:syn1} shows various gray levels that require an adaptive algorithm. \subref{fig:syn2}
    shows more challenging edge detection tests that have crossing lines. Fusing these into full segments typically requires algorithms like the Hough transform.
    This is an example of using subfigures, with \texttt{subref}s in the caption.
    }\label{fig:synthetic}
\end{figure}

\clearpage

\subsection{Equations}

Equations should be typeset correctly and precisely. Make sure you get parenthesis sizing correct, and punctuate equations correctly 
(the comma is important and goes \textit{inside} the equation block). Explain any symbols used clearly if not defined earlier. 

For example, we might define:
\begin{equation}
    \hat{f}(\xi) = \frac{1}{2}\left[ \int_{-\infty}^{\infty} f(x) e^{2\pi i x \xi} \right],
\end{equation}    
where $\hat{f}(\xi)$ is the Fourier transform of the time domain signal $f(x)$.

\subsection{Algorithms}
Algorithms can be set using \texttt{algorithm2e}, as in Algorithm \ref{alg:metropolis}.

% NOTE: line ends are denoted by \; in algorithm2e
\begin{algorithm}
    \DontPrintSemicolon
    \KwData{$f_X(x)$, a probability density function returing the density at $x$.\; $\sigma$ a standard deviation specifying the spread of the proposal distribution.\;
    $x_0$, an initial starting condition.}
    \KwResult{$s=[x_1, x_2, \dots, x_n]$, $n$ samples approximately drawn from a distribution with PDF $f_X(x)$.}
    \Begin{
        $s \longleftarrow []$\;
        $p \longleftarrow f_X(x)$\;
        $i \longleftarrow 0$\;
        \While{$i < n$}
        {
            $x^\prime \longleftarrow \mathcal{N}(x, \sigma^2)$\;
            $p^\prime \longleftarrow f_X(x^\prime)$\;
            $a \longleftarrow \frac{p^\prime}{p}$\;
            $r \longleftarrow U(0,1)$\;
            \If{$r<a$}
            {
                $x \longleftarrow x^\prime$\;
                $p \longleftarrow f_X(x)$\;
                $i \longleftarrow i+1$\;
                append $x$ to $s$\;
            }
        }
    }
    
\caption{The Metropolis-Hastings MCMC algorithm for drawing samples from arbitrary probability distributions, 
specialised for normal proposal distributions $q(x^\prime|x) = \mathcal{N}(x, \sigma^2)$. The symmetry of the normal distribution means the acceptance rule takes the simplified form.}\label{alg:metropolis}
\end{algorithm}

\subsection{Tables}

If you need to include tables, like Table \ref{tab:operators}, use a tool like https://www.tablesgenerator.com/ to generate the table as it is
extremely tedious otherwise. 

\begin{table}[]
    \caption{The standard table of operators in Python, along with their functional equivalents from the \texttt{operator} package. Note that table
    captions go above the table, not below. Do not add additional rules/lines to tables. }\label{tab:operators}
    %\tt 
    \rowcolors{2}{}{gray!3}
    \begin{tabular}{@{}lll@{}}
    %\toprule
    \textbf{Operation}    & \textbf{Syntax}                & \textbf{Function}                            \\ %\midrule % optional rule for header
    Addition              & \texttt{a + b}                          & \texttt{add(a, b)}                                    \\
    Concatenation         & \texttt{seq1 + seq2}                    & \texttt{concat(seq1, seq2)}                           \\
    Containment Test      & \texttt{obj in seq}                     & \texttt{contains(seq, obj)}                           \\
    Division              & \texttt{a / b}                          & \texttt{div(a, b) }  \\
    Division              & \texttt{a / b}                          & \texttt{truediv(a, b) } \\
    Division              & \texttt{a // b}                         & \texttt{floordiv(a, b)}                               \\
    Bitwise And           & \texttt{a \& b}                         & \texttt{and\_(a, b)}                                  \\
    Bitwise Exclusive Or  & \texttt{a \textasciicircum b}           & \texttt{xor(a, b)}                                    \\
    Bitwise Inversion     & \texttt{$\sim$a}                        & \texttt{invert(a)}                                    \\
    Bitwise Or            & \texttt{a | b}                          & \texttt{or\_(a, b)}                                   \\
    Exponentiation        & \texttt{a ** b}                         & \texttt{pow(a, b)}                                    \\
    Identity              & \texttt{a is b}                         & \texttt{is\_(a, b)}                                   \\
    Identity              & \texttt{a is not b}                     & \texttt{is\_not(a, b)}                                \\
    Indexed Assignment    & \texttt{obj{[}k{]} = v}                 & \texttt{setitem(obj, k, v)}                           \\
    Indexed Deletion      & \texttt{del obj{[}k{]}}                 & \texttt{delitem(obj, k)}                              \\
    Indexing              & \texttt{obj{[}k{]}}                     & \texttt{getitem(obj, k)}                              \\
    Left Shift            & \texttt{a \textless{}\textless b}       & \texttt{lshift(a, b)}                                 \\
    Modulo                & \texttt{a \% b}                         & \texttt{mod(a, b)}                                    \\
    Multiplication        & \texttt{a * b}                          & \texttt{mul(a, b)}                                    \\
    Negation (Arithmetic) & \texttt{- a}                            & \texttt{neg(a)}                                       \\
    Negation (Logical)    & \texttt{not a}                          & \texttt{not\_(a)}                                     \\
    Positive              & \texttt{+ a}                            & \texttt{pos(a)}                                       \\
    Right Shift           & \texttt{a \textgreater{}\textgreater b} & \texttt{rshift(a, b)}                                 \\
    Sequence Repetition   & \texttt{seq * i}                        & \texttt{repeat(seq, i)}                               \\
    Slice Assignment      & \texttt{seq{[}i:j{]} = values}          & \texttt{setitem(seq, slice(i, j), values)}            \\
    Slice Deletion        & \texttt{del seq{[}i:j{]}}               & \texttt{delitem(seq, slice(i, j))}                    \\
    Slicing               & \texttt{seq{[}i:j{]}}                   & \texttt{getitem(seq, slice(i, j))}                    \\
    String Formatting     & \texttt{s \% obj}                       & \texttt{mod(s, obj)}                                  \\
    Subtraction           & \texttt{a - b}                          & \texttt{sub(a, b)}                                    \\
    Truth Test            & \texttt{obj}                            & \texttt{truth(obj)}                                   \\
    Ordering              & \texttt{a \textless b}                  & \texttt{lt(a, b)}                                     \\
    Ordering              & \texttt{a \textless{}= b}               & \texttt{le(a, b)}                                     \\
    % \bottomrule
    \end{tabular}
    \end{table}
\subsection{Code}

Avoid putting large blocks of code in the report (more than a page in one block, for example). Use syntax highlighting if possible, as in Listing \ref{lst:callahan}.

\begin{lstlisting}[language=python, float, caption={The algorithm for packing the $3\times 3$ outer-totalistic binary CA successor rule into a 
    $16\times 16\times 16\times 16$ 4 bit lookup table, running an equivalent, notionally 16-state $2\times 2$ CA.}, label=lst:callahan]
    def create_callahan_table(rule="b3s23"):
        """Generate the lookup table for the cells."""        
        s_table = np.zeros((16, 16, 16, 16), dtype=np.uint8)
        birth, survive = parse_rule(rule)

        # generate all 16 bit strings
        for iv in range(65536):
            bv = [(iv >> z) & 1 for z in range(16)]
            a, b, c, d, e, f, g, h, i, j, k, l, m, n, o, p = bv

            # compute next state of the inner 2x2
            nw = apply_rule(f, a, b, c, e, g, i, j, k)
            ne = apply_rule(g, b, c, d, f, h, j, k, l)
            sw = apply_rule(j, e, f, g, i, k, m, n, o)
            se = apply_rule(k, f, g, h, j, l, n, o, p)

            # compute the index of this 4x4
            nw_code = a | (b << 1) | (e << 2) | (f << 3)
            ne_code = c | (d << 1) | (g << 2) | (h << 3)
            sw_code = i | (j << 1) | (m << 2) | (n << 3)
            se_code = k | (l << 1) | (o << 2) | (p << 3)

            # compute the state for the 2x2
            next_code = nw | (ne << 1) | (sw << 2) | (se << 3)

            # get the 4x4 index, and write into the table
            s_table[nw_code, ne_code, sw_code, se_code] = next_code

        return s_table

\end{lstlisting}

%==================================================================================================================================
\chapter{Evaluation} 
How good is your solution? How well did you solve the general problem, and what evidence do you have to support that?

\section{Guidance}
\begin{itemize}
    \item
        Ask specific questions that address the general problem.
    \item
        Answer them with precise evidence (graphs, numbers, statistical
        analysis, qualitative analysis).
    \item
        Be fair and be scientific.
    \item
        The key thing is to show that you know how to evaluate your work, not
        that your work is the most amazing product ever.
\end{itemize}

\section{Evidence}
Make sure you present your evidence well. Use appropriate visualisations, reporting techniques and statistical analysis, as appropriate.

If you visualise, follow the basic rules, as illustrated in Figure \ref{fig:boxplot}:
\begin{itemize}
\item Label everything correctly (axis, title, units).
\item Caption thoroughly.
\item Reference in text.
\item \textbf{Include appropriate display of uncertainty (e.g. error bars, Box plot)}
\item Minimize clutter.
\end{itemize}

See the file \texttt{guide\_to\_visualising.pdf} for further information and guidance.

\begin{figure}
    \centering
    \includegraphics[width=1.0\linewidth]{images/boxplot_finger_distance.pdf}    

    \caption{Average number of fingers detected by the touch sensor at different heights above the surface, averaged over all gestures. Dashed lines indicate
    the true number of fingers present. The Box plots include bootstrapped uncertainty notches for the median. It is clear that the device is biased toward 
    undercounting fingers, particularly at higher $z$ distances.
    }

    % use the notation fig:name to cross reference a figure
    \label{fig:boxplot} 
\end{figure}


%==================================================================================================================================
\chapter{Conclusion}    
Summarise the whole project for a lazy reader who didn't read the rest (e.g. a prize-awarding committee).
\section{Guidance}
\begin{itemize}
    \item
        Summarise briefly and fairly.
    \item
        You should be addressing the general problem you introduced in the
        Introduction.        
    \item
        Include summary of concrete results (``the new compiler ran 2x
        faster'')
    \item
        Indicate what future work could be done, but remember: \textbf{you
        won't get credit for things you haven't done}.
\end{itemize}

%==================================================================================================================================
%
% 
%==================================================================================================================================
%  APPENDICES  

\begin{appendices}

\chapter{Appendices}

Typical inclusions in the appendices are:

\begin{itemize}
\item
  Copies of ethics approvals (required if obtained)
\item
  Copies of questionnaires etc. used to gather data from subjects.
\item
  Extensive tables or figures that are too bulky to fit in the main body of
  the report, particularly ones that are repetitive and summarised in the body.

\item Outline of the source code (e.g. directory structure), or other architecture documentation like class diagrams.

\item User manuals, and any guides to starting/running the software.

\end{itemize}

\textbf{Don't include your source code in the appendices}. It will be
submitted separately.

\end{appendices}

%==================================================================================================================================
%   BIBLIOGRAPHY   

% The bibliography style is abbrvnat
% The bibliography always appears last, after the appendices.

\bibliographystyle{abbrvnat}

\bibliography{l4proj}

\end{document}
